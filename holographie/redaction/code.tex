\begin{minted}
% bf_traitement
%
% TP Formation de voies
% Jean-Claude Pascal et Jean-Hugh Thomas
%
close all; clear all;
%-- détermination de la méthode : 'bartlet','capon','music'
bfmethod = 'music';

disp(' '), disp(['-- Traitement avec la méthode ',bfmethod]), disp(' ')

%-- construction du path
rootpath = cd;
addpath([rootpath '/Bf_bib']),

%==============================================================================
% Lecture de la matrice interspectrale  (code complet)
%------------------------------------------------------------------------------

%-- Sélection du fichier hdf
%
FilterSpec = {'*.hdf','hdf file'};
[HDFFileName,HDFPathName] = uigetfile(FilterSpec,'load an hdf spectral file');
        
if ischar(HDFFileName)
    filename = [HDFPathName HDFFileName];
else
    return,
end

%-- Visualisation du spectre moyen sur les microphones de l'antenne
%
[avspect,freqvect,axename] = HDFinterface('averageddata',filename);   
HDFinterface('averageddata',filename);

%-- Sélection de la fréquence de traitement
%
disp(' '), disp('Sélection de la fréquence'),
freq = input('     entrer la fréquence à traiter (en Hz) : ');
[freq,ifreq] = nearest(freqvect,freq);

%-- Chargement de la matrice interspectrale
%
[Refarray,axevector,axeorder] = HDFfileAScontrol('getdata',filename,'Refarray',ifreq,':',1);
Gpp = TransRefarray('vec2mat',Refarray,'single');     
Gpp = conj(Gpp);

% INFO : Gpp = Gpp'; dans Matlab Gpp' représente la transposée hermitienne de la matrice complexe Gpp
%        la matrice Gpp est une matrice carrée [M M] (M nombre de microphones)

%==============================================================================
% Lecture des coordonnées des points de l'antenne (code complet)
%------------------------------------------------------------------------------
%
% INFO
% micropnts est une matrice [M 3] où M est le nombre de microphones de l'antenne
% micropnts(:,1) est le vecteur colonne des coordonnées x
% micropnts(:,2) est le vecteur colonne des coordonnées y
% micropnts(:,3) est le vecteur colonne des coordonnées z (normalement nul car le 
% plan de l'antenne est en z = 0)

[micropnts,coordsys,arraysys] = HDFinterface('micropnts',filename);

%==============================================================================
% Détermination du plan de représentation  (code à compléter : donner des valeurs) 
%------------------------------------------------------------------------------

% INFO
% Le plan de représentation est parallèle à celui de l'antenne Pour définir les points 
% où seront estimées les sources il faut fournir les informations suivantes :
%  Nx, Ny -> le nombre de points en x et y
%  dist   -> la distance du plan de représentation à celui de l'antenne
%  Xmin Xmax Ymin Ymax -> les limites du plan de représentation 
% L'origine du repère est située sur l'axe de l'antenne. Par exemple :
Nx = 40;
Ny = 40;
dist = 1.5;
Xmin = -1;
Xmax = 1;
Ymin = -1;
Ymax = 1;
M=size(micropnts);
M=M(1);

%-- construction du maillage sur le plan de représentation
%
x = linspace(Xmin,Xmax,Nx);
y = linspace(Ymin,Ymax,Ny);
[Xmat,Ymat] = meshgrid(x,y);

%-- vecteur [Np 3] des positions des sources
Np = Nx*Ny;
srcpnts = [Xmat(:) Ymat(:) -dist*ones(Np,1)];

%==============================================================================
% Pré-traitement selon la méthode choisie  (code à compléter) 
%------------------------------------------------------------------------------
if strcmpi(bfmethod,'capon')
    disp('   :: pré-traitement pour la méthode de Capon'),
    % INFO
    % le traitement consiste ici à inverser la matrice Gpp
    
    Gpp_inv=Gpp^-1;

elseif strcmpi(bfmethod,'music')
    disp('   :: pré-traitement pour la méthode de MUSIC'),
    % INFO
    % le traitement consiste ici à décomposer la matrice Gpp en utilisant la fonction svd
    % de Matlab [U,S,V] = svd(Gpp)  (dans ce cas particulier V = U')

    [u,s,v] = svd(Gpp);


end    
    
%==============================================================================
% Boucle de traitement pour chacun des points sources  (code à compléter) 
%------------------------------------------------------------------------------
hw = waitbar(0,['traitement méthode ',bfmethod,' ...']);
S = zeros(Np,1);

for ii=1:Np
    
    waitbar(ii/Np,hw);
    
    %-- vecteur [1 3] des coordonnées du point source
    coorsrc = srcpnts(ii,:);
    
    %--------------------------------------------------------------------------
    % Calcul des distances 
    % calcul du vecteur R [M,1] des distance entre chaque microphone et le point 
    % source
    %--------------------------------------------------------------------------
    
    for m=1:M
       R(m)=norm(coorsrc-micropnts(m,:));
    end
    
    
    %--------------------------------------------------------------------------
    % Calcul du vecteur h [M,1] représentant les fonctions de transfert entre 
    % les microphones et le point source (Eqs. 2.2 et 2.3)
    %--------------------------------------------------------------------------
    c=343; %célérité du son dans l'air en m/s
    k=2*pi*freq/c;
    
    h=exp(-j*k*R)./(4*pi*R);
    h=h';
    
    
    %--------------------------------------------------------------------------
    % Calcul du vecteur de pilotage selon la méthode 
    % La méthode MUSIC n'est pas concernée par cette phase
    %--------------------------------------------------------------------------

    if strcmpi(bfmethod,'bartlet')
        % INFO
        % voir Eq. 3.7
    
        w=(h'*h)^(-1)*h;

    elseif strcmpi(bfmethod,'capon')
        % voir Eq. 5.5
    
        w=Gpp_inv*h/(h'*Gpp_inv*h);


    end    
    
    %--------------------------------------------------------------------------
    % Calcul de la distribution des sources selon les méthodes
    % les résultats du calcul sont rangés dans un vecteur S [Np 1]
    %--------------------------------------------------------------------------
    
    if strcmpi(bfmethod,'bartlet')
        % INFO
        % voir Eq. 2.5
    
        S(ii)=w'*Gpp*w; %dsp
        

    elseif strcmpi(bfmethod,'capon')
        % voir Eq. 2.5
    
        S(ii)=w'*Gpp*w; %dsp
        

    elseif strcmpi(bfmethod,'music')
        % voir Eq. 6.3
        somme=0;
        q0=15; %taille de l'espace signal
        for q=q0:M
            somme= somme+ (abs(h'*u(:,q)))^2;
        end
        S(ii)=1/somme; %Pi
         
    end    
    
%------------------------------------------------------------------------------
end % fin de la boucle ii=1:Np

close(hw),
%------------------------------------------------------------------------------


%==============================================================================
% Reconstitution de la matrice rectangulaire et visualisation (code à compléter) 
%------------------------------------------------------------------------------
%
% INFO
% selon meshgrid S doit avoir comme dimensions [Ny Nx]
S = reshape(S,Ny,Nx);
S = real(S);

%-- utiliser ici éventuellement une interpolation pour avoir un maillage de 
%   représentation plus fin  (fonction interp2 de Matlab)
x1 = x;
y1 = y;
S1 = S;

% A COMPLETER EVENTUELLEMENT

%-- visualiser la carte des sources en utilisant la fonction imagesc
%

titre = ['Méthode ',bfmethod,' - f = ',num2str(freq),' Hz'];





reptype = 'lin';  % 'lin' ou 'dB'

if strcmpi(reptype,'lin')
    
%-- pour une représentation linéaire
repstruct.mode = 'mod*';
repstruct.dynscal = [];   %  -> range of representation of scalar map (used for dB)
repstruct.maxscal = [];    %  -> max value of scalar map ( [] -> automatic scaling)
repstruct.stepscal = 10;   %  -> step for rounded max value in automatic scaling
repstruct.dBref = 1;       %  -> energy reference for dB (A = 10 log [real(Z)/dBref])
repstruct.title = titre;   %  -> map title string
repstruct.underrangecolor = [0.9 0.9 0.9]; %  -> underrange color
repstruct.unit = '';       %  -> string of quantity unit to put under the colorbar

else
    
%-- pour une représentation en dB
Dyn = 15;
RefdB = 1e-12;
repstruct.mode = 'dB*';
repstruct.dynscal = Dyn;   %  -> range of representation of scalar map (used for dB)
%repstruct.maxscal = dBmax; %  -> max value of scalar map ( [] -> automatic scaling)
repstruct.maxscal = [];   %  -> max value of scalar map ( [] -> automatic scaling)
repstruct.stepscal = 1;    %  -> step for rounded max value in automatic scaling
repstruct.dBref = RefdB;   %  -> energy reference for dB (A = 10 log [real(Z)/dBref])
repstruct.title = titre;   %  -> map title string
repstruct.underrangecolor = [0.95 0.95 0.95]; %  -> underrange color
repstruct.unit = 'dB';     %  -> string of quantity unit to put under the colorbar

end

figure
%imagesc(x1,y1,S1)
ccmap(repstruct,x1,y1,S1);



\end{minted}
